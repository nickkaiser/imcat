% notes regarding determination of geometry of a mosaic camera a la Luppino

\documentstyle{article}
\begin{document}

\section{MOSAIC CAMERA GEOMETRY FITTING}

\subsection{The Problem}

We assume that the telescope produces a distorted image of
the sky on the detector plane, with some simple parameterised
form for distortion, and that the CCD chips are roughly
positioned on a regular grid, but with small offsets and
rotations.  Given a set of dithered observations we want to
determine
\begin{itemize}
\item parameters of the distortion model
\item shifts and rotations of the chips relative to asumed grid
\item global shift and linear transformations for dithered fields
\end{itemize}
The method of solution described here is
\begin{enumerate}
\item Convert pixel positions to `nominal' detector frame coords
(i.e.~assuming chips are perfectly positioned on the grid).
\item Guess parameters for chip rotations, offsets and distortion (e.g~all
zero).
\item \label{item:makesuperlists} For each exposure, from the
lists of star coordinates from the individual chips
generate a `superlist' containing
approximate sky positions determined using guessed parameters
and chip ID. 
\item For a pair of exposures, solve approximately for relative shift,
rotation of 2nd exposure.
\item Transform 2nd superlist; merge superlists (with some coarse
tolerance); and untransform coordinates from 2nd exposure.
\item Un-apply any non trivial transformation in step
\ref{item:makesuperlists} so we're back to nominal chip coords.
\item Using pairs of these coordinates from merged superlists solve for
all parameters in a linearised manner.
\end{enumerate}
Now using these improved parameters we can
\begin{enumerate}
\item \label{item:transform} Apply transformation from
nominal coords to sky coords using current parameters
(including shift, rotation to bring second exposure coords onto first).
\item Merge lists using fine tolerance
\item Apply inverse of tranformation in step \ref{item:transform}.
\item Solve for all parameters using improved merged list.
\end{enumerate}
and iterate if necessary.

In fact we have more than one pair of pointings so we really want to solve for
everything simultaneously using all pairings of objects....

The goal is that with these tranformations we can map the individual
images from each exposure onto a single image so we can then
process a stack of such images to do cosmic ray removal and
averaging.

\subsection{Mathematical formulation}

The starting point is a set of catalogues for $\sim 50$ or so stars
per chip per exposure giving integer pixel coords $i_x, i_y$. 
Let chips have nominal size $N_x \times N_y$ ($2048 \times 4096$).

Our first task is to generate a first approximation to
rectilinear sky coords based on assumtions re size, spacing
and orientation of chips.  Call these coordinates `nominal
detector coords'.  We generate these as follows:

First we make transformation for upper row of chips (which are
`upside down':
\begin{equation}
\matrix{
i_x \rightarrow N_x - i_x - 1 \cr
i_y \rightarrow N_y - i_y - 1
}
\end{equation}

We then model chips as mounted on blocks of size $(N_x + 2 M_x) \times
(N_y + 2 M_y)$ so $M_x$, $M_y$ are the margin widths.
We assign chips an integer position $(I_x,I_y)$
where $I_x = (-2,...,1)$, $I_y = (-1,0)$). (Bottom left chip
is (-2,-1); chip-5 has readout pixel at centre of grid and
has position $(0,0)$.


Generate `nominal detector coords' for $c$th chip and $e$th exposure
\begin{equation}
\left[\matrix{
x_{ce} \cr y_{ce}
}\right]
=
\left[\matrix{
I_x * (N_x + 2 M_x) + M_x + i_x 
\cr
I_y * (N_y + 2 M_y) + M_y + i_y
}\right]
\end{equation}
from the pixel positions $i,j$ and chip position.
We assume these are related to perfect rectilinear detector
coords $(x,y)$ by
\begin{equation}
\left[\matrix{
x_e \cr y_e
}\right]
=
\left[\matrix{
1 & \phi_c
\cr
-\phi_c & 1
}\right]
\left[\matrix{
x_{ce}
\cr
y_{ce}
}\right]
+
\left[\matrix{
dx_c
\cr
dy_c
}\right]
\end{equation}

We assume that perfect detector coords are related to 
locally rectilinear sky coords by
\begin{equation}
\left[\matrix{
X_e
\cr
Y_e
}\right]
=
(1 + \alpha (x_e^2 + y_e^2))
\left[\matrix{
x_e
\cr
y_e
}\right]
\end{equation}
for some fiducial telescope pointing and orientation, and that
absolute sky coordinates
$(X,Y)$ are given by
\begin{equation}
\left[\matrix{
X
\cr
Y
}\right]
=
\left[\matrix{
X_e
\cr
Y_e
}\right] + 
\left[\matrix{
\Phi_e^{00} & \Phi_e^{01}
\cr
\Phi_e^{10} & \Phi_e^{11}
}\right]
\left[\matrix{
X_e
\cr
Y_e
}\right]
+
\left[\matrix{
dX_e
\cr
dY_e
}\right]
\end{equation}

Let chip number $c$ run from $0$ to $N_c - 1$ and let exposure
number $e$ run from $0$ to $N_e - 1$
We set $\phi_0 = dx_0 = dx_0 = \Phi_0 = dX_0 = dY_0 = 0$, and linearise
in the $1 + 3 (N_e - 1) + 3 (N_c - 1)$
assumed very small parameters $\alpha$; $\Phi_m,dX_m,dY_m$, 
$m = 1\ldots N_e -1$; and $\phi_n,dx_n,dy_n$, $n = 1 \ldots N_c -1$
to obtain
\begin{equation}
\left[\matrix{
X
\cr
Y
}\right]
=
\left[\matrix{
x_{ce}
\cr
y_{ce}
}\right]
+
\alpha (x_{ce}^2 + y_{ce}^2)
\left[\matrix{
x_{ce}
\cr
y_{ce}
}\right]
+
\left[\matrix{
\Phi_e^{00} & \Phi_e^{01}
\cr
\Phi_e^{10} & \Phi_e^{11}
}\right]
\left[\matrix{
x_{ce}
\cr
y_{ce}
}\right]
+
\left[\matrix{
dX_e
\cr
dY_e
}\right]
+
\phi_c
\left[\matrix{
y_{ce}
\cr
-x_{ce}
}\right]
+
\left[\matrix{
dx_c
\cr
dy_c
}\right]
\end{equation}
To determine coefficients we need to generate lists of pairs of
positions $(X,Y)$ $(X',Y')$ for same object and minimise
\begin{equation}
\chi^2 = \sum (\Delta X^2 + \Delta Y^2)
\end{equation}
where $\Delta X \equiv X - X'$ etc.

Packaging the variables in a vector $V_I$ we can write
\begin{equation}
\left[\matrix{
\Delta X
\cr
\Delta Y
}\right]
=
\left[\matrix{
x_{ce} - x_{c'e'}
\cr
y_{ce} - y_{c'e'}
}\right]
+
\left[\matrix{
C^x_I V_I
\cr
C^y_I V_I
}\right]
\end{equation}
where
\begin{equation}
V_I = 
\left[\matrix{
\alpha \cr
\vdots \cr
\Phi_m^{00} \cr
\vdots \cr
\Phi_m^{01} \cr
\vdots \cr
\Phi_m^{10} \cr
\vdots \cr
\Phi_m^{11} \cr
\vdots \cr
dX_m \cr
\vdots \cr
dY_m \cr
\vdots \cr
\phi_n \cr
\vdots \cr
dx_n \cr
\vdots \cr
dy_n \cr
\vdots
}\right]
\quad\quad
C^x_I =
\left[\matrix{
r^2 x - r'^2 x' \cr
\vdots \cr
x \delta_{me} - x' \delta_{me'} \cr
\vdots \cr
y \delta_{me} - y' \delta_{me'} \cr
\vdots \cr
0 \cr
\vdots \cr
0 \cr
\vdots \cr
\delta_{me} - \delta_{me'}\cr
\vdots \cr
0 \cr
\vdots \cr
y \delta_{nc} - y' \delta_{nc'} \cr
\vdots \cr
\delta_{nc} - \delta_{nc'} \cr
\vdots \cr
0 \cr
\vdots
}\right]
\quad\quad
C^y_I =
\left[\matrix{
r^2 y - r'^2 y' \cr
\vdots \cr
0 \cr
\vdots \cr
0 \cr
\vdots \cr
x \delta_{me} - x' \delta_{me'} \cr
\vdots \cr
y \delta_{me} - y' \delta_{me'} \cr
\vdots \cr
0 \cr
\vdots \cr
\delta_{me} - \delta_{me'} \cr
\vdots \cr
-x \delta_{nc} + x' \delta_{nc'} \cr
\vdots \cr
0 \cr
\vdots \cr
\delta_{nc} - \delta_{nc'}  \cr
\vdots
}\right]
\end{equation}
and we now have 
\begin{equation}
\chi^2 = \sum (\Delta x + C^x_I V_I)^2 + (\Delta y + C^y_I V_I)^2
\end{equation}
where $\Delta x \equiv x_{ce} - x_{c'e'}$ and so minimising wrt the
variable $V_I$ yields the set of linear equations:
\begin{equation}
A_{IJ} V_J = B_J
\end{equation}
where
\begin{equation}
A_{IJ} = \sum C^x_I C^x_J + C^y_I C^y_J
\quad\quad\quad\quad
B_J = -\sum C^x_J \Delta x + C^y_J \Delta y
\end{equation}



\subsection{Databases and software}

The parameters for the nominal chip size, margin etc are kept in
{\tt nominal.db}.  (All these databases are in perl format as they get
read by perl scripts).

Chip names, integer positions and orientation of chips are stored
as associative arrays in {\tt chips.db}.  The key is an integer
which runs from $0\ldots N_c - 1$.  Orientation is 1 for lower
row of chips, -1 for upper.

Field names (for a1413 run these were exposure numbers `031',
`032' etc.) are stored as associative array in {\tt fields.db}.

Assume that one has generated 
{\tt basedir/chip\$chipname/\$fieldname.stars}
cats of moderately bright stars
(I used
{\tt select -m 16 19 -rx 0.8 1.2} to select them) then
you first want to run {\tt superlists.pl} which generates
a set of $N_e$ files  (1 per exposure) containing 
nominal detector coords $x,y$; chip number $c$, and exposure
number $e$.

The next step is to run {\tt makemergelist.pl} which figures
out an approximate rotation and translation to register pairs of
superlists (we use all pairings of exposures) and generate a
file {\tt mergelists.out} containing $x,y,c,e,x',y',c',e'$.

We now feed {\tt mergelists.out} to {\tt mosaicfit -c 7 -e 11}
which solves for $\alpha$; $\Phi_e, dX_e, dY_e$; $\phi_c, dx_c, dy_c$
and redirect output into {\tt mosaicfit.par}.

We now run {\tt makemergelist2.pl} which uses 1st approximation
to paramaters in {\tt mosaicfit.par} make a refined
merged list: {\tt mergelists2.out} which we feed back to
{\tt mosaicfit -c 7 -e 11} (perhaps after filtering on
column 9 which contains residual displacement) to give
a refined set of transformation parameters.

For the 11 exposures and 7 chips for a1413 data the result was
\begin{verbatim}
 -1.92821e-10
            0             0             0
 -6.01328e-05        296.92       10.5084
    0.0004332       271.737        1202.7
  0.000364594      -262.278       1218.76
  0.000282352      -380.615       1090.64
  0.000300921       1105.92       1159.14
  0.000401059       1265.68       1321.57
   0.00897166      -748.315       1151.47
   0.00893781        -161.3       1188.75
   0.00881633      -147.714        897.66
   0.00940282       731.594       935.408
            0             0             0
  0.000571406       9.25005       8.26747
   0.00255388       9.46212       7.78085
   0.00169522       2.99181        10.856
   0.00245214        6.8493       0.62552
 -0.000119338      -5.15558      -12.0497
  0.000498863      -1.47967      -10.6429
\end{verbatim}

The first line contains distortion coefficient $\alpha$.

The next 11 lines contain $\Phi_e, dX_e, dY_e$ (zero for
1st exposure).

The next 7 lines contain $\phi_c, dx_c, dy_c$ (zero for
first chip).

Now we can generate a stack of images for a chosen section
of $X,Y$ space using {\tt makestack.pl} which figures out
which chips lie under the chosen section for each exposure and 
warps them using {\tt mosaicmap}.

The final step is to combine the images (median, avsigclip
or whatever) with {\tt pastiche}.

\end{document}
